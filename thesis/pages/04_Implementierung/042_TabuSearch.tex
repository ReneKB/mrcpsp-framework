\subsection{Tabu Search} \label{subsec:MetaheuristischeAlgorithmen_TabuSearch}

Bereits im Abschnitt \ref{subsec:Grundlagen_TabuSearch} wurde die Tabu Suche als eine Erweiterung des naiven \ac{LS}-Algorithmus (bzw. Hill Climbing) eingeführt. Wesentliche Komponenten, wie die generelle Funktionsweise der schrittweisen Verbesserungen, die Nachbarschaftsfunktion oder das Erzeugen der initialen Zeitpläne bleiben gleich. Folglich wurden die Implementierungen der Komponenten für die \ac{TS} aus Abschnitt \ref{subsec:Hill_Climbing} übernommen. Die Implementierung der Tabu Suche wurde in der Klasse \lstinline|TabuSearchSolver| realisiert. \\

Eins der am meist verbreitetsten (Basis-)Konzepte für die Tabu Search stellt die Tabu List dar \cite[vgl.][S. 42]{gendreau_handbook_2019}. Dieses wurde im Rahmen der eigenen Implementation aufgegriffen. Die Größe der Tabu List stellt einen Hyperparameter dar, welcher mit $|TL| = \sqrt{|J| - 2}$ versehen wurde. Des Weiteren werden neue Elemente am Anfang der Liste hinzugefügt. Die restlichen Elemente sind anschließend jeweils um eine Position verschoben, wobei das letzte Element von der Liste entfernt wird, sofern dieses nicht mehr in die Liste passt. Elemente aus der Tabu Liste werden gemäß des Konzepts nicht in der Nachbarschaft in Betracht gezogen \cite[vgl.][S. 42]{gendreau_handbook_2019}. 