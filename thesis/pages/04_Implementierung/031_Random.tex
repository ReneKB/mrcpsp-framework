\subsection{Zufällige Lösungen} \label{subsec:Naiv_Random}

Den einfachsten Lösungsansatz stellt das Generieren von zufälligen Lösungen dar. Der \lstinline|RandomSolver| generiert mit Hilfe des \ac{ssgs} und dem \lstinline|HeuristisDirector| zunächst syntaktisch korrekte Zeitpläne. Hierbei werden als Prioritäts- und Selektionsregeln die Klassen \lstinline|RandomActvitiyHeuristic| und  \lstinline|RandomModeHeuristic| eingesetzt, welche zufällige Zahlen aus dem Intervall $[0, 10000]$ generieren. Innerhalb des Algorithmus wird in jeder Iteration überprüft, ob der gefundene Zeitplan primär die Makespan $C_{max}$ und ggf. sekundär die Robustheitsfunktion $\Omega$ verbessert und legt den besten Zeitplan fest. Dieser Algorithmus wird über eine endliche Zahl an Iterationen durchlaufen. 