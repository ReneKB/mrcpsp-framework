\chapter{Konzept des MRCPSP-Frameworks} \label{ch:Konzept}

Dieses Kapitel befasst sich mit der Konzeption eines MRCPSP-Frameworks, welches zur Beantwortung der Forschungsfrage und dessen Unterfragen maßgeblich ist. Hierfür wird im Abschnitt \ref{sec:Konzeptüberblick} zunächst ein Überblick der Anforderungen gegeben, welche zu implementieren sind. Basierend auf den Anforderungen werden im selben Abschnitt die Teilaspekte des Frameworks hergeleitet. Konzeptionsentscheidungen, wie die Auswahl der Verfahren für den Umgang von Unsicherheitsentscheidungen (Abschnitt \ref{sec:AuswahlVerfahrenUnsicherheiten}) und die Auswahl der Metaheuristiken (Abschnitt \ref{sec:AuswahlMetaheuristischenAlgorithmen}) werden ebenfalls im Kapitel behandelt. Um Vergleiche auf eine gemeinsame Datenbasis zu realisieren, befasst sich Abschnitt \ref{sec:Benchmarkdatensatz} mit der Auswahl des Benchmarkdatensatzes. Die Konzeption von Unsicherheitsszenarien wird im Abschnitt \ref{sec:Unsicherheitsszenarien} behandelt. 

% \section{Problemstellung} \label{sec:Problemstellung}
