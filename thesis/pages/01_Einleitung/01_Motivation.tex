\chapter{Einleitung} \label{ch:Einleitung}

Dieses Kapitel der Einführung beginnt zunächst mit dem Abschnitt \ref{sec:Motivation}, welcher das Thema der Masterarbeit motiviert. Im Folgeabschnitt werden die anzustrebenden Ziele der Masterarbeit einschließlich der Forschungsfrage und die dazugehörigen Unterfragen vorgestellt. Der Aufbau der vorliegenden Arbeit wird anschließend im Abschnitt \ref{sec:Aufbau} erläutert.

\section{Motivation} \label{sec:Motivation}

\glqq Plans are worthless, but planning is everything!\grqq{} \cite[S. 235]{eisenhower_dwight_1958} Dwight D. Eisenhower vom 14. November 1957. In dem Zitat vom 34. Präsidenten der USA steckt viel Bedeutung, denn diese Aussage kann dahingehend interpretiert werden, dass in Plänen Ereignisse und Eventualitäten nicht stets berücksichtigt werden. Zu den häufigen Fehlern innerhalb des Projektmanagements gehören Fehlentscheidungen, unklare Abhängigkeiten, Ressourcenknappheiten und vieles weitere \cite[vgl.][S. 74]{nelson_it_2007}. Die Folgen von solchen Fehlern sind nicht zuletzt Verspätungen innerhalb des Zeitplans.  \\

Die ressourcenbeschränkte Projektplanung ist ein Feld aus dem Bereich Operation Management und befasst sich mit der Planung von Projekten mit Ressourcen. Ein Projekt besteht aus einzelnen Aktivitäten, welche wiederum Ressourcen zur Vollendung benötigen. Ein Projekt kann beispielsweise die Entwicklung eines neuen Produktes sein, das Bauen eines Gebäudes, aber auch schon das Backen eines Kuchens. Projekte mit Ressourcenbeschränkungen sind bereits zudem in vielen Unternehmen allgegenwärtig. \cite[vgl.][S. 1 f.]{kellenbrink_ressourcenbeschrankte_2014} \\

Zeitpläne ohne Ressourcenbeschränkungen können über Vor- und Rückwärtskal-kulationen erstellt werden \cite[vgl.][S. 9 f.]{kellenbrink_ressourcenbeschrankte_2014}. Jedoch stellen die Beschränkungen bei der ressourcenbeschränkten Projektplanung ein Problem dar, da die Ressourcen innerhalb der Aktivitätsausführung durchgehend zur Verfügung stehen müssen, womit andere Verfahren, wie (Meta-)Heuristiken benötigt werden, um gültige Zeitpläne zu erstellen \cite[vgl.][S. 11]{kellenbrink_ressourcenbeschrankte_2014}. Dies stellt das ressourcenbeschränkte Projektplanungsproblem (RCPSP) dar \cite[vgl.][S. 11]{kellenbrink_ressourcenbeschrankte_2014}, welches als ein NP-hartes Optimierungsproblem angesehen wird \cite[vgl.][S. 2]{kolisch_heuristic_1998}. Im Mittelpunkt der Masterarbeit steht eine Problemerweiterung des RCPSP, nämlich das Multimodus ressourcenbeschränkte Projektplanungsproblem (MRCPSP) welches Aktivitäten über eine von mehreren zur Verfügung stehenden Alternativen durchlaufen werden muss \cite[vgl.][S. 596]{wuliang_improved_2014}. \\

Der Umgang mit Unsicherheiten, sei es aufgrund einer Fehlplanung, tatsächlicher Ressourcenknappheiten, Krankheiten innerhalb des Personals oder weiteren Gründen stellt ebenfalls einen wesentlicher Faktor der Masterarbeit dar. Hierfür existieren verschiedene Verfahren, wie die proaktive, prädiktive oder reaktive Planung. Bei der proaktiven Planung werden Unsicherheiten ignoriert. Der prädiktive Ansatz nutzt weitere Kennziffern, wie die Robustheit eines Zeitplans, welche es zu maximieren gilt, um so Verspätungen durch Unsicherheiten zu verringern. Beim reaktiven Ansatz wird zum Zeitpunkt einer Unsicherheit reagiert, um so den Zeitplan zu reparieren oder einen neuen Zeitplan zu erstellen. \cite[vgl.][S. 404 f.]{brcic_resource_2012} \\

Zum Zeitpunkt der Masterarbeit bietet das Multi-Mode Ressource Constrained Project Scheduling Problem (MRCPSP) mit dem Umgang von Unsicherheiten weiterhin ein hohes Forschungspotential. Ein direkter Vergleich von Verfahren für den Umgang mit Unsicherheiten wurde für das MRCPSP noch nicht gezogen. Die Kombination von Metaheuristiken mit den pro-, prä-, reaktiven Methoden gilt es zudem ebenfalls zu evaluieren. 