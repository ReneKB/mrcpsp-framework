\section{Aufbau der Arbeit} \label{sec:Aufbau}

Zu Beginn der Arbeit wurde bereits das Thema der Masterarbeit motiviert und die Ziele innerhalb der Problemstellung definiert. Die folgenden Kapitel der Arbeit befassen sich mit den folgenden Inhalten:
\begin{description}
    \item[Kapitel 2] In dem Kapitel gilt es die theoretischen Grundlagen des (Multi-Mode) \acl{rcpsp} zu erläutern. Sowohl das Grundproblem als auch die Multi-Mode Problemerweiterung werden zunächst formal definiert. Zudem sollen heuristische Lösungsansätze für das \acs{mrcpsp} vorgestellt werden. Des Weiteren werden in dem Kapitel die theoretischen Grundlagen von einer Auswahl an Metaheuristiken vorgestellt. Zuletzt wird in dem Kapitel der aktuelle Stand der Forschung präsentiert, in welcher das Thema der Masterarbeit eingeordnet wird. \\ 
    \item[Kapitel 3] Das darauffolgende Kapitel befasst sich mit der Konzeption des MRCPSP-Framework. Hierbei wird im ersten Abschnitt gemäß der Problemstellung basierend auf dem Stand der Forschung ein Konzeptüberblick gegeben. Relevante Teilkomponenten des Konzepts werden im weiteren Verlauf des Kapitels behandelt. \\
    \item[Kapitel 4] Dieses Kapitel basiert auf dem im vorherigen Kapitel vorgestellten Konzept und geht auf die konkrete Umsetzung des MRCPSP-Framework ein. Einleitend wird zunächst die Struktur des Frameworks vorgestellt. Im Anschluss werden einzelne Teilkomponenten erläutert, welche für die Beantwortung der Forschungsfragen relevant sind. \\
    \item[Kapitel 5] Wesentlicher Bestandteil des Kapitels ist der quantitative Vergleich der entwickelten pre-, prä- und reaktiven Verfahren und Metaheuristiken gemäß der Problemstellung. \\ 
    \item[Kapitel 6] Das abschließende Kapitel \ref{ch:Zusammenfassung}, bestehend aus dem Fazit und dem Ausblick, beantwortet die Forschungsfrage und dessen Unterfragen anhand der Evaluation und fasst die Ergebnisse der Arbeit zusammen. Themen und Problemstellungen für weiterführende Arbeiten sind dem Ausblick zu entnehmen. \\ 
\end{description}