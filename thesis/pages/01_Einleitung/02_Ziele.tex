\section{Problemstellung} \label{sec:Ziele}

Das Problem der Unsicherheiten gilt es für das \ac{mrcpsp} zu lösen, um so die einhergehenden Verspätungen so gering wie möglich zu halten. Im Rahmen der Masterarbeit wird daher die folgende Forschungsfrage aufgestellt: Welche Auswirkungen haben prädiktive und reaktive Ansätze für das MRCPSP auf Basis von metaheuristischen Algorithmen bei der Erstellung von Zeitplänen in Bezug auf die Projektdauer bei Verspätungen? \\

Die Forschungsfrage wird mithilfe weiterer Unterfragen gestützt. Welche prädiktiven und reaktiven Ansätze können für das MRCPSP in Kombination mit metaheuristischen Algorithmen angewandt werden? Lässt sich ein Vergleich zwischen prädiktiven und reaktiven Ansätzen auf das Ergebnis in Bezug zur Projektdauer ziehen? Welchen Einfluss haben hierbei Metaheuristiken zur Findung von \glqq guten\grqq{} Lösungen? \\

Um die Unterfragen beantworten zu können müssen die proaktiven, prädiktiven und reaktiven Verfahren auf Benchmarks mit Unsicherheitsszenarien angewandt werden. Diese Benchmarks müssen zunächst für das MRCPSP selektiert werden. Zudem gilt es die Unsicherheitsszenarien innerhalb der Benchmarks zu konzipieren und zu integrieren. Um geeignete Lösungen für das Problem zu finden sollen Metaheuristiken eingesetzt werden. Die Auswahl und die Integrierung, aber auch deren Evaluierung der Metaheuristiken über einen Vergleich der gefundenen Projektdauern sind ebenfalls wesentlicher Bestandteil dieser Arbeit. 