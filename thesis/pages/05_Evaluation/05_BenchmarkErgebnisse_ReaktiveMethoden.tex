\section{Benchmark-Ergebnisse der reaktiven Methoden} \label{sec:BenchmarkErgebnisse_ReaktiveMethoden}

Auf der identischen Datenbasis zu Abschnitt \ref{sec:BenchmarkErgebnisse_NaivMethoden} und \ref{sec:BenchmarkErgebnisse_PraediktiveMethoden} werden in diesem Abschnitt die Ergebnisse der reaktiven Methode vorgestellt. Hierbei werden zunächst Basiszeitpläne gefunden, welche die Makespan $C_{max}$ minimieren. Bei einem Verspätungszeitpunkt wird anschließend ein Folgezeitplan über eine Tabu Search gesucht, welcher die Kostenfunktion $\mathcal{C}$ aus Abschnitt \ref{subsec:Reaktive_Methoden} minimiert. Der Algorithmus nutzt als initiale Lösung den aktuellen Zeitplan und durchläuft 500 Iterationen. Die Ergebnisse der Methode für das Instanzset n1 sind in der Tabelle \ref{tab:evaluation_reactive_n1} visualisiert. Weitere Ergebnisse sind aus dem Anhang \ref{sec:WeitereAuswertung_ProaktivReaktiv} zu entnehmen. 

{\footnotesize
\begin{longtable}{ll|rrrr|rrrr}
\toprule
\textbf{Instance set n1}                 & {} & \multicolumn{4}{c|}{Mean $\mu$} & \multicolumn{4}{c}{Std. Dev $\sigma$} \\
                & Uncertainty $p$: & 0\% & 5\% & 10\% & 20\% & 0\% & 5\% & 10\% & 20\% \\
Solver & Iteration &      &      &      &      &      &      &      &      \\
\midrule
RandomSolver & 500  & 6.27 & 6.61 & 6.97 & 7.74 & 3.51 & 3.56 & 3.63 & 3.88 \\
                 & 1000 & 5.64 & 6.03 & 6.40 & 7.16 & 3.32 & 3.40 & 3.48 & 3.52 \\
                 & 2500 & 4.69 & 5.08 & 5.52 & 6.37 & 2.95 & 3.00 & 3.08 & 3.22 \\
                 & 5000 & 3.73 & 4.21 & 4.66 & 5.65 & 2.45 & 2.53 & 2.68 & 2.85 \\ \hline
HillClimbing & 500  & 3.82 & 4.39 & 4.93 & 5.98 & 3.51 & 3.47 & 3.44 & 3.41 \\
                 & 1000 & 3.84 & 4.44 & 4.93 & 6.04 & 3.38 & 3.41 & 3.41 & 3.39 \\
                 & 2500 & 3.64 & 4.21 & 4.76 & 5.84 & 3.13 & 3.15 & 3.18 & 3.18 \\
                 & 5000 & 3.94 & 4.54 & 5.10 & 6.05 & 4.06 & 4.09 & 4.11 & 3.96 \\ \hline
TabuSearch & 500  & 1.77 & 2.45 & 3.17 & 4.44 & 2.20 & 2.29 & 2.39 & 2.65 \\
                 & 1000 & 1.21 & 1.96 & 2.68 & 4.00 & 1.81 & 2.01 & 2.17 & 2.28 \\
                 & 2500 & 0.98 & 1.76 & 2.53 & 3.94 & 1.93 & 2.13 & 2.26 & 2.50 \\
                 & 5000 & 0.50 & 1.36 & 2.10 & 3.64 & 0.99 & 1.40 & 1.65 & 2.04 \\ \hline
SimulatedAnnealing & 500  & 2.91 & 3.51 & 4.02 & 5.21 & 2.55 & 2.59 & 2.68 & 2.78 \\
                 & 1000 & 1.81 & 2.52 & 3.17 & 4.42 & 2.14 & 2.33 & 2.44 & 2.70 \\
                 & 2500 & 1.05 & 1.81 & 2.54 & 3.86 & 1.41 & 1.73 & 1.90 & 2.20 \\
                 & 5000 & 0.67 & 1.51 & 2.27 & 3.67 & 1.05 & 1.48 & 1.76 & 2.02 \\ \hline
GeneticAlgorithm & 500  & 2.17 & 2.78 & 3.36 & 4.52 & 2.12 & 2.11 & 2.20 & 2.26 \\
                 & 1000 & 1.15 & 1.85 & 2.50 & 3.90 & 1.42 & 1.58 & 1.71 & 2.06 \\
                 & 2500 & 0.55 & 1.39 & 2.16 & 3.56 & 0.89 & 1.38 & 1.62 & 2.06 \\
                 & 5000 & 0.49 & 1.32 & 2.10 & 3.53 & 0.87 & 1.39 & 1.62 & 2.02 \\
\bottomrule
\caption{Vergleich der reaktiven Lösungsverfahren auf $m=50$ unterschiedliche Unsicherheitsszenarien für das Instanzset n1. }
\label{tab:evaluation_reactive_n1}
\end{longtable}
}
\vspace*{-25px}
\begin{figure}[H]
\source{Eigene Darstellung}
\end{figure}