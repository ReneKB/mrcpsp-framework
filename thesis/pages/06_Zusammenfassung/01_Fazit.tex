\chapter{Zusammenfassung und Ausblick} \label{ch:Zusammenfassung}
Der abschließende Teil dieser Masterarbeit besteht aus einem Fazit und einem Ausblick. Im Abschnitt \ref{sec:Fazit} werden die Kernthemen der Masterarbeit zusammengefasst und die wichtigsten Ergebnisse vorgestellt um die Forschungsfrage und dessen Unterfragen zu beantworten. Eine Eingliederung innerhalb des Forschungsstands wird mit einem Ausblick im Abschnitt \ref{sec:Ausblick} aufgeführt.

\section{Fazit} \label{sec:Fazit}

Das Thema der Masterarbeit basiert auf dem \ac{mrcpsp}, welches als NP-vollständig klassifiziert wurde. In dieser Arbeit wurde für das Problem ein Framework konzipiert und implementiert, welches Zeitpläne anhand von Aktivitäts- und Moduslisten erzeugt. Diese wiederum werden über eine Vielzahl an Lösungsvarianten erzeugt, um so die Zeitpläne mit der kürzesten Projektdauer $C_{max}$ zu finden. Insbesondere die implementierten Metaheuristiken \acl{TS}, \acl{SA} und \acl{GA} dienen dazu, Lösungen zu finden, welche das globale Optimum annähern. Initiale Zeitpläne wurden über das \ac{ssgs} und etabilierte heuristische Aktivitäts- und Selektionsregeln erzeugt. Die im Forschungsfeld anerkannte PSPLIB beinhaltet unterschiedliche Benchmark Sets für das \ac{mrcpsp}, die über das Framework geladen werden können. \\

Unsicherheiten, wie Aktivitätsstörungen führen zu Verspätungen bei den Projektplänen. Unsicherheitsszenarien bestehen aus Mengen von Aktivitätsstörungen, die mit einer Binomialverteilung erzeugt werden konnten. Zur Minimierung der Verspätungen wurden pro-, prä- und reaktive Verfahren konzipiert und implementiert. Um die Forschungsfrage zu beantworten, galt es diese Verfahren auf Benchmark Sets und eine Menge an zufallsgenerierten Unsicherheitsszenarien zu evaluieren. Das proaktive Verfahren und das Ignorieren von Verspätungen, dient hierbei als Referenzverfahren. Bei dem prädiktiven Verfahren wurde bereits im Suchprozess über die Metaheuristiken eine weitere Zielfunktion, nämlich die Robustheit $\Omega$ berücksichtigt, welche es zu maximieren gilt. Hierfür wurden zunächst unterschiedliche etablierte Robustheitsmessungsfunktionen selektiert und implementiert. Diese galt es anschließend auf Benchmark Sets und Unsicherheitsszenarien quantitativ zu vergleichen. Als reaktives Verfahren wurde das Reparieren von Zeitplänen zu den Unsicherheitszeitpunkten ausgewählt. Für die Reparatur wurde eine erweiterte Tabu Suche entwickelt, welche mit einer festen Anzahl an Iterationen zum Zeitpunkt reagiert. Diese minimiert die Kostenfunktion $\mathcal{C}$. Somit galt es zur Laufzeit Zeitpläne zu finden, die sich nah am Basiszeitplan orientieren und trotzdem die Verspätung zur geplanten $C_{max}$ minimiert. \\

Die Forschungsfrage befasste sich mit der Evaluierung der Auswirkungen von prädiktiven und reaktiven Verfahren auf die Projektdauer bei Verspätungen im \ac{mrcpsp}. Mithilfe eines quantitativen Vergleiches zwischen den implementierten Repräsentanten beider Verfahren gilt es die Forschungsfrage zu beantworten. Die Auswertung zeigte auf, dass sich beide Verfahren für die Verminderung der Projektverspätungen eignen. Der Grad des Unsicherheitslevels ist maßgeblich für die konkrete Auswahl des Verfahrens. Bei einem geringen Unsicherheitslevel zeigte im Schnitt das prädiktive Verfahren über die Robustheitsoptimierung eine bessere Wirkung als das reaktive Verfahren. Umgekehrt zeigte das reaktive Verfahren bei einem hohen Unsicherheitslevel eine bessere Wirkung, als das prädiktive Verfahren. Ein direkter Vergleich ist dennoch schwierig, da beim reaktiven Ansatz die Berechnungsdauer signifikant höher als beim prädiktiven Ansatz ist. Zudem eignen sich reaktive Verfahren bei Basiszeitplänen, die mit einer geringen Anzahl an Iterationen gefunden wurden. Hierbei pendeln sich über die Reparaturvorgänge womöglich bessere Zeitpläne ein und profitieren somit stark von der Güte des Reparaturalgorithmus. Reaktive Verfahren können in der Praxis insbesondere für flexible Projekte genutzt werden, während prädiktive Verfahren die Pläne über den Puffer aufrecht erhalten. Unabhängig der prädiktiven und reaktiven Verfahren stellen Metaheuristiken wesentliche Techniken zur Findung von adäquaten Basiszeitplänen dar. Es zeigte sich zudem, dass die Unterschiede zwischen den Lösungsverfahren und durchlaufenden Iterationen sowohl bei den Basiszeitplänen als auch bei den verspäteten Zeitplänen in der Relation betrachtet weiterhin vorhanden sind. 

% Welche Auswirkungen haben prädiktive und reaktive Ansätze für das MRCPSP auf Basis von metaheuristischen Algorithmen bei der Erstellung von Zeitplänen in Bezug auf die Projektdauer bei Verspätungen?

% Die Forschungsfrage wird Mithilfe weitere Unterfragen gestützt. Welche prädiktive und reaktive Ansätze können für das MRCPSP in Kombination mit metaheuristischen Algorithmen angewandt werden? Lässt sich ein Vergleich zwischen prädiktiven und reaktiven Ansätzen auf das Ergebnis in Bezug zur Projektdauer ziehen? Welcher Einfluss haben hierbei Metaheuristiken zur Findung von \glqq guten\grqq{} Lösungen?
