\chapter*{Abstract (deutsch)}
\thispagestyle{empty}
Das \acf{mrcpsp} stellt ein NP-vollständiges Optimierungsproblem dar, welches sich mit der Erzeugung von Projektplänen mit Ressourcenbeschränkungen befasst. Unsicherheiten, wie z. B. Verspätungen führen dazu, dass ein Projektplan nicht mehr in geplanter Zeit vollendet werden kann. \\

Das Ziel der Masterarbeit befasst sich zunächst mit der Suche von Basiszeitplänen für Projekte mit Hilfe von Metaheuristiken, welche die Projektdauer minimieren sollen. Anschließend gilt es prädiktive und reaktive Verfahren zu selektieren um die einhergehenden Verspätungen durch Unsicherheiten zu minimieren. \glqq Welche Auswirkungen haben prädiktive und reaktive Ansätze für das MRCPSP auf Basis von metaheuristischen Algorithmen bei der Erstellung von Zeitplänen in Bezug auf die Projektdauer bei Verspätungen?\grqq{} stellt die Forschungsfrage der Thesis. \\

Diese Forschungsfrage wurde über eine quantitative Studie mit Hilfe eines zu implementierenden Framework beantwortet. Hierfür werden unterschiedliche Metaheuristiken konzipiert und implementiert. Als prädiktives Verfahren wurde die Robustheitsoptimierung und als reaktives Verfahren die Reparatur eines Zeitplans über eine Kostenfunktion zum Unsicherheitszeitpunkt ausgewählt. Über das Durchlaufen der pro-, prä- und reaktiven Verfahren auf eine gemeinsame Benchmark-Basis und auf eine Vielzahl an zufällig generierten Unsicherheitsszenarien konnten Daten für die quantitative Analyse erhoben werden. \\

Die Daten für \acf{mrcpsp} zeigten, dass beide Verfahren Verspätungen durch Aktivitätsstörungen signifikant minimieren, wobei das prädiktive Verfahren insbesondere für kleinere und das reaktive Verfahren für komplexere Unsicherheitsszenarien geeignet ist. 

\chapter*{Abstract (english)}
\thispagestyle{empty}
The \acf{mrcpsp} represents an NP-complete optimization problem dealing with the generation of project plans with resource constraints. Uncertainties, such as delays, cause a project plan to fail to complete in planned time. \\

The goal of the master thesis is first to find baseline schedules for projects using metaheuristics, which should minimize the project duration. Subsequently, predictive and reactive methods have to be selected in order to minimize the associated delays due to uncertainties. \glqq What is the impact of predictive and reactive approaches for MRCPSP based on metaheuristic algorithms in generating schedules in terms of project duration in the presence of delays?\grqq{} poses the research question of the thesis. \\

This research question was answered via a quantitative study using a framework that needed to be implemented. For this purpose, different metaheuristics are designed and implemented. Robustness optimization was selected as the predictive procedure and repair of a schedule via a cost function at the uncertainty time as the reactive procedure. By running the proactive, preactive and reactive procedures on a common benchmark basis and on a variety of randomly generated uncertainty scenarios, data for quantitative analysis could be obtained. \\

The data of the \acs{mrcpsp} showed that both methods significantly minimize delays due to activity perturbations, with the predictive method being particularly suitable for smaller uncertainty scenarios and the reactive method for more complex scenarios. 